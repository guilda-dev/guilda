\documentclass[landscape]{report}

% Set page size and margins
\pagestyle{empty}
\usepackage[letterpaper,top=2cm,bottom=2cm,left=3cm,right=3cm,margin=15truemm]{geometry}

% Useful packages
\usepackage{amsmath}
\usepackage{graphicx}
\usepackage{autobreak}
\usepackage{longtable}

\title{Information in PowerNetwork}
\author{by GUILDA}

\begin{document}
\maketitle{}
\section{ネットワークのグラフ構造}

 \begin{figure}[h]
 \begin{center}
 \includegraphics[width=0.9\linewidth]{data/network_graph.eps}
 \end{center}
 \end{figure}
 \newpage

\section{母線の情報(潮流状態・接続機器の種類)}
\input{data/bus.txt}

\section{送電線のパラメータ}
\input{data/brnch.txt}

\section{機器一覧}
\subsection{機器のパラメータ}
\input{data/component_parameter.txt}

\subsection{機器の平衡点}
\input{data/component_equilibrium.txt}

\subsection{各機器のダイナミクス}
\input{data/component_dynamics.txt}

\end{document}